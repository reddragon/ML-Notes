\documentclass{article}
\usepackage[utf8]{inputenc}
\usepackage[T1]{fontenc}    % use 8-bit T1 fonts
\usepackage{hyperref}       % hyperlinks
\usepackage{url}            % simple URL typesetting
\usepackage{booktabs}       % professional-quality tables
\usepackage{amsmath}
\usepackage{mathtools}
\usepackage{amsfonts}       % blackboard math symbols
\usepackage{nicefrac}       % compact symbols for 1/2, etc.
\usepackage{microtype}      % microtypography
\usepackage{lipsum}
\usepackage{graphicx}
\usepackage{graphbox}
\usepackage{subfig}
\usepackage[ruled,vlined,linesnumbered]{algorithm2e}
\usepackage{algorithmic}
\usepackage{multirow}
\usepackage{caption}
\usepackage{array}
\usepackage{pgfplots}
\usepackage{dsfont}
\newcommand{\somecomment}[1]{}
\newcommand\ddfrac[2]{\frac{\displaystyle #1}{\displaystyle #2}}
\captionsetup[table]{skip=10pt}
\newcolumntype{L}{>{\centering\arraybackslash}m{3cm}}
\DeclareMathOperator*{\argmax}{argmax}
\DeclareMathOperator*{\argmin}{argmin}
\newcommand\mbf{\mathbf}

\title{Machine Learning Notes}
\author{Gaurav Menghani}
\date{April 2022}

\begin{document}

\maketitle

These are mostly notes for concepts that I keep forgetting about. So keeping a doc of running notes to ensure I remember them.

\section{Empirical Risk Minimization}
Assuming there is a hypothesis $h \in H$, s.t. $h: X \to Y$. There is also a risk function $R(h) = \mathbb{E} [L(h(x), y)]$. We want to choose $h^*$ such that it minimizes $R(h)$ for all possible $h \in H$, i.e.:

\begin{equation}
\begin{split}
    h^*& = \argmin_{h \in H} R(h)\\
    & = \argmin_{h \in H} \mathbb{E} [L(h(x), y)]\\
    & = \int_{x, y} L(h(x), y) dP(x, y)
\end{split}
\end{equation}

Since we do not know the true joint distribution $P(x, y)$, we must empirically estimate it using the training dataset of size $n$ examples.

In this case we can construct an \textit{empirical} risk $R_{\rm emp}(h)$, s.t.

\begin{equation}
    R_{\rm emp} = \ddfrac{1}{n} \sum_{i=1}^{n} L(h(x_i), y_i)
\end{equation}

The hypothesis $\hat{h}$ which mininimizes this empirical risk, is what we are looking for in ERM (TODO: Cite Vapnik1991).

\begin{equation}
    \hat{h} = \argmin_{h \in H} R_{\rm emp}
\end{equation}

\section{Maximum Likelihood Estimation (MLE)}

\section{Maximum A Posteriori (MAP) Estimate}

\section{Prediction v/s Inference}
For the prediction task we are trying to estimate $P(y | x)$. This is useful in settings like classification.

For the inference task we are trying to estimate $P (x | y)$.

\section{Bias Variance Tradeoff}
\begin{equation}
\begin{split}
    \mathbb{E}[f(x, y) - \hat{f}(x, y)]^2 &= \rm{Variance\ Terms} + \rm{Bias\ Terms} + \rm{Irreducible\ Error}.
\end{split}
\end{equation}

Variance terms are more if the function $f$ is flexible, i.e., changing the $n$ datapoints used for training will lead to a very different $f$. This happens when there is overfitting.

Bias terms are dominant when the function $f$ is inflexible, i.e., $\mathbb{E}[f(x, y) - \hat{f}(x, y)]$ is large. This happens when there is underfitting.

\section{Summation Bounds}
Bernstein Bounds, Jensen Inequality, Chernoff Bounds, Hoeffdling's Inequality.

\section{Bayes Classifier and Naive Bayes Classifier}

\section{Parameteric v/s Non-Parametric Models}
Parametric models need a lot of parameters, where as non-parameteric models need a lot of data points. A neural network is a parametric model, and a k-Nearest Neighbor model is a non-parametric model.

\section{k-Nearest Neighbors (kNN)}
This is a non-parametric model where we can infer the label $y$ of a given $x$ based on the neighbors of $x$.

\section{k-Means Clustering}

\section{Irreducible Error}
For many problems, the response $\mbf y$ can be represented in terms of the predictors $\mbf x$ as follows:

\begin{equation}
\mbf{y} = f(\mbf{x}) + \epsilon
\end{equation}

Where $\epsilon$ is the irreducible error, and $f$ is the function that we learnt. The reason $\epsilon$ could be > 0 because of noise, missing predictors, etc.

\section{Flexible and Inflexible Functions}
Consider a function $f$.

\section{Precision and Recall}
In terms of TP, FP, TN, FN.

\section{AUC}
TODO: How is AUC computed.

\section{Online Learning}

\section{Backpropagation}
TODO: An example about backprop.

\section{Stochastic Gradient Descent}
Some theory about gradient descent, and why does it have to be stochastic.

\section{Minibatches}
Why are minibatches required for deep learning?

\section{Active Learning}
Goals of active learning are to find useful and diverse examples to label.

\section{Simulated Annealing}
One of the probabilistic techniques for approximating the global optimum of a given function $f$. There are others such as Genetic Algorithms, Ant Colony Optimization, etc.

It starts with a temperature $T$ and a starting state $s_0$, and continues to find neighbors that might be better, while gradually reducing the temperature. This continues to happen until a given point when $T = 0$.

\section{Neural Tangent Kernel}

\bibliographystyle{ACM-Reference-Format}
\bibliography{bib}
\end{document}
